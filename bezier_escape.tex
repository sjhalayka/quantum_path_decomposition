\documentclass[12pt]{article}

\title{On the quantum decomposition of the planet Mercury's orbit path}
\author{S. Halayka\footnote{sjhalayka@gmail.com}}
\date{\today\;\currenttime}

\usepackage{datetime}
\usepackage{listings}
\usepackage{cite}
\usepackage{xcolor}
\usepackage{graphicx}
\usepackage{setspace}
\usepackage{amsmath}
\usepackage{url}
\usepackage[margin=1.0in]{geometry}

%\doublespace

%\usepackage[]{lineno}
%\linenumbers


\begin{document}



 
\maketitle

\begin{abstract}
By quantizing the gravitational time dilation using various step sizes, one obtains a set of weighted paths.
The precession associated with each weighted path combines to provide the same answer as the classical analytical solution.
\end{abstract}


\section{Time dilation}

The kinematic time dilation is:
\begin{equation}
\label{eq_intro_kinematic}
\frac{d\tau}{dt} = \frac{\sqrt{c^2 - \lvert\lvert \vec{v}\rvert\rvert^2}}{c} = \sqrt{1 - \frac{\lvert\lvert \vec{v}\rvert\rvert^2}{c^2}}.
\end{equation}

The gravitational time dilation is:
\begin{equation}
\label{eq_intro_gravitational}
\frac{d\tau}{dt} = \sqrt{1 - \frac{R_s}{r}}.
\end{equation}

In this paper we will be quantizing the kinematic and gravitational time dilation by casting them to a smaller floating point number. The quantized step size (e.g. epsilon) in general is $\epsilon = 2^{-m}$ where $m$ is the number of mantissa bits in the floating point number.






\section{Steps in spacetime}

Where $\ell_s$ denotes the Sun's location at the origin, $\ell_o$ denotes the orbiter's location, and $\vec{d}$ denotes the direction vector that points from the orbiter toward the Sun:
\begin{equation}
\label{direction_vector}
\vec{d} = \ell_{s} - \ell_{o},	
\end{equation}
\begin{equation}
\label{direction_unit_vector}
\hat{d} = \frac{\vec{d}}{\lvert\lvert \vec{d} \rvert\rvert},
\end{equation}
the Newtonian acceleration vector is:
\begin{equation}
\label{newton}
\vec{g}_n = \frac{\hat{d} G M}{{\lvert\lvert \vec{d} \rvert\rvert}^2}.
\end{equation}

One parameter is closely related to the kinematic time dilation:
\begin{equation}
\label{eq_kinematic}
\alpha = 2 - \sqrt{1 - \frac{\lvert\lvert \vec{v}_{o}\rvert\rvert^2}{c^2}}.
\end{equation}
Another parameter is the gravitational time dilation:
\begin{equation}
\label{eq_gravitational}
\beta = \sqrt{1 - \frac{R_{s}}{\lvert \lvert \vec{d} \rvert \rvert}}.
\end{equation}

Finally, the semi-implicit Euler integration for velocity and then location is:
\begin{align}
\label{eq_velocity}
\vec{v}_{o}(t + \delta_t) &= \vec{v}_{o}(t) + \delta_{t} \alpha \vec{g}_n, \\
\label{eq_position}
\ell_{o}(t + \delta_t) &= \ell_{o}(t) + \delta_{t} \beta \vec{v}_{o}(t + \delta_t).
\end{align}
Note that Newtonian gravity is the result where $\alpha = \beta = 1$.


\section{Precession}

\begin{equation}
\label{delta_p}
\delta_{p} = \frac{6 \pi G M}{c^2 (1 - e^2) a} \left( \frac{1}{ \pi \times 180 \times 3600} \right) \left( \frac{365}{88} \times 100 \right) = 42.937
\end{equation}




\


\begin{figure} 
\centering
\label{fig4}
  \includegraphics[width = 4 in]{precession.png}
  \caption{ A diagram showing precession, where the orbit does not quite form a closed ellipse.
}
\end{figure}



\begin{thebibliography}{9}


\bibitem{halayka} Halayka. On simulating the four Solar System tests of general relativity using two-parameter post-Newtonian gravitation with Euler integration. (2024)

\end{thebibliography}





\end{document}









